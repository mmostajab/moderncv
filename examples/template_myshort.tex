%% start of file `template.tex'.
%% Copyright 2006-2015 Xavier Danaux (xdanaux@gmail.com).
%
% This work may be distributed and/or modified under the
% conditions of the LaTeX Project Public License version 1.3c,
% available at http://www.latex-project.org/lppl/.


\documentclass[11pt,a4paper,sans]{moderncv}        % possible options include font size ('10pt', '11pt' and '12pt'), paper size ('a4paper', 'letterpaper', 'a5paper', 'legalpaper', 'executivepaper' and 'landscape') and font family ('sans' and 'roman')

% moderncv themes
\moderncvstyle{casual}                             % style options are 'casual' (default), 'classic', 'banking', 'oldstyle' and 'fancy'
\moderncvcolor{blue}                               % color options 'black', 'blue' (default), 'burgundy', 'green', 'grey', 'orange', 'purple' and 'red'
%\renewcommand{\familydefault}{\sfdefault}         % to set the default font; use '\sfdefault' for the default sans serif font, '\rmdefault' for the default roman one, or any tex font name
%\nopagenumbers{}                                  % uncomment to suppress automatic page numbering for CVs longer than one page

% character encoding
%\usepackage[utf8]{inputenc}                       % if you are not using xelatex ou lualatex, replace by the encoding you are using
%\usepackage{CJKutf8}                              % if you need to use CJK to typeset your resume in Chinese, Japanese or Korean

% adjust the page margins
\usepackage[scale=0.75]{geometry}
%\setlength{\hintscolumnwidth}{3cm}                % if you want to change the width of the column with the dates
%\setlength{\makecvheadnamewidth}{10cm}            % for the 'classic' style, if you want to force the width allocated to your name and avoid line breaks. be careful though, the length is normally calculated to avoid any overlap with your personal info; use this at your own typographical risks...

% personal data
\name{Morteza}{Mostajab}
%\title{}                               % optional, remove / comment the line if not wanted
\address{Kasinostr. 24}{64293 Darmstadt}{Germany}% optional, remove / comment the line if not wanted; the "postcode city" and "country" arguments can be omitted or provided empty
\phone[mobile]{+49~(170)~548~5750}                   % optional, remove / comment the line if not wanted; the optional "type" of the phone can be "mobile" (default), "fixed" or "fax"
\email{mostajab@in.tum.de}                               % optional, remove / comment the line if not wanted
%\homepage{www.johndoe.com}                         % optional, remove / comment the line if not wanted
%\social[linkedin]{john.doe}                        % optional, remove / comment the line if not wanted
\social[twitter]{https://twitter.com/mmostajab}                             % optional, remove / comment the line if not wanted
\social[github]{https://github.com/mmostajab}                              % optional, remove / comment the line if not wanted
%\extrainfo{additional information}                 % optional, remove / comment the line if not wanted
\photo[64pt][0.4pt]{mypic.jpg}                       % optional, remove / comment the line if not wanted; '64pt' is the height the picture must be resized to, 0.4pt is the thickness of the frame around it (put it to 0pt for no frame) and 'picture' is the name of the picture file
\quote{computer graphics and visualization fan and researcher}                                 % optional, remove / comment the line if not wanted

% bibliography adjustements (only useful if you make citations in your resume, or print a list of publications using BibTeX)
%   to show numerical labels in the bibliography (default is to show no labels)
\makeatletter\renewcommand*{\bibliographyitemlabel}{\@biblabel{\arabic{enumiv}}}\makeatother
%   to redefine the bibliography heading string ("Publications")
%\renewcommand{\refname}{Articles}

% bibliography with mutiple entries
%\usepackage{multibib}
%\newcites{book,misc}{{Books},{Others}}
%----------------------------------------------------------------------------------
%            content
%----------------------------------------------------------------------------------
\begin{document}
%\begin{CJK*}{UTF8}{gbsn}                          % to typeset your resume in Chinese using CJK
%-----       resume       ---------------------------------------------------------
\makecvtitle

\section{Education}
\cventry{2012--2016}{Master of computer science}{Technische Universit\"at M\"unchen}{Munich}{}{Specialization: Computer graphics and visualization}  % arguments 3 to 6 can be left empty
\cventry{2006--2011}{Bacholer of computer engineering}{Hamedan University of Technology}{Hamedan, Iran}{Major: Computer hardware engineering}{}
%{Grade: \textit{15.04} out of 20.0}
\cventry{2005--2006}{Pre-university}{National Organization for Development of Exceptional Talents' Shahid Beheshti School}{Borujerd, Iran}{}{Major: Mathematics and physics}
\cventry{2002--2005}{High school}{National Organization for Development of Exceptional Talents' Shahid Beheshti School}{Borujerd, Iran}{}{Major: Mathematics and physics}

\section{Research Interests}
\cvitem{}{Rendering Techniques (Ray tracing and Rasterization)}
\cvitem{}{Virtual Reality}
\cvitem{}{SciVis Techniques}
\cvitem{}{Computer Graphics and Visualization }
\cvitem{}{Object Oriented Programming (OOP)}

\section{Master thesis}
\cvitem{title}{\emph{Real-time streamsurface computation and rendering}}
\cvitem{supervisors}{Prof.Dr. Westermann}
\cvitem{advisors}{Dr. Andreas Dietrich, Dr. Frank Michel}
%\cvitem{abstract}{Streamsurfaces are one of the powerful visualization tools, which are used to gain insight into characteristics and features of flow fields. In practice, streamsurfaces are approximated by triangulating adjacent pairs of integral curves, originating from a seeding line. The generation of integral curves bears quite some similarities to ray tracing algorithms used in physically based renderers. Although, the techniques used in ray tracing may not have good performance in the streamline computation context due to their different computational nature, they can be optimized for streamline computation by introducing some modifications. In this master thesis, I present my work on accurate streamsurface computation and rendering in real-time, by exploiting the scalability and portability features of parallel architectures in heterogeneous computing, and utilizing concepts from physically based rendering. To improve the efficiency, I use a scheduler to divide the streamsurface computation and rendering tasks on different devices proportional to their computation powers. Additionally, I apply and evaluate different acceleration structures and the concepts of caching to improve the efficiency and utilization of streamsurface generation on modern GPUs and CPUs to achieve real-time results. Furthermore, the possible impact of applying ray-packing and ray-sorting to the streamline computation is investigated.}

\section{Bachelor thesis}
\cvitem{title}{\emph{Incorporating affective states of players in video games}}
\cvitem{supervisor}{Dr. Muharram Mansoorizadeh}
%\cvitem{abstract}{Video games are very popular, these days. They use different ways to interact with users. Keyboards, mouse, Joysticks, and recently camera are among the popular interaction devices. While technology involved in game industry develops new interactive devices to get user's interactions that capture lots of player's inputs, the games just use player's voluntary interactions. However, spontaneous behaviors of players are of much value in playing games, the players do not use involuntary interactions. Among several available representations for affects, two-dimensional continuous Activation/Evaluation(Valence) space has been adopted. We can determine the player's preferences in the environment by comparing his or her playing to the player's valence and activity in real-time. These information can be used to modify the game rules and environment to make it more attractive for players and create a unique experience during each gameplay session. (Score:20/20).}

\section{Employment}

\cventry{2016--Present}{Researcher}{Fraunhofer IGD}{Darmstadt}{}{Computer graphics research and developments. \newline{} %
Projects:%
         \begin{itemize}
	\item VELaSSco (Visualization For Extremely Large-Scale Scientific Computing) EC project development (\href{htpp://www.velassco.eu}{VELaSSco.eu}). 
         \end{itemize}}

\section{Publications}

\cventry{2016}{Real-time Stream Surface Computation and Rendering}{Master Thesis}{}{}{Computer graphics research and developments.\newline{}%
	Detailed achievements:%
	\begin{itemize}%
		\item participating in VELaSSCo EC project development;
		\item Higher-order primitive ray-tracer implemented in Intel Embree and NVIDIA OptiX.
		\item Virtual reality development with LEAP Motion and Oculus SDK.
	\end{itemize}}

\cventry{2011}{Incorporating affective state of players in video games}{Bachelor Thesis}{}{}{Computer graphics research and developments.\newline{}%
	Detailed achievements:%
	\begin{itemize}%
		\item participating in VELaSSCo EC project development;
		\item Higher-order primitive ray-tracer implemented in Intel Embree and NVIDIA OptiX.
		\item Virtual reality development with LEAP Motion and Oculus SDK.
	\end{itemize}}

\section{Teaching}

\cventry{2016}{Seminar Course Supervision}{Technische Universitaet Darmstadt}{Germany}{}{Detailed achievements:%
	\begin{itemize}%
		\item Teaching assistant, B.S. Introduction to Programming, M.Sc. Hassan Bashiri, spring 2008.
		\item Teaching assistant, B.S. Advanced Programming, M.Sc. Hassan Bashiri, autumn 2008.
		\item Teaching assistant, B.S. Introduction to Assembly 80x86 Programming, M.Sc. Hatam Abdoli, spring 2009.
		\item Teaching assistant, B.S. Data Structures, Dr. Mir Hossein Dezfoulian, autumn 2009.
		\item Teaching assistant, B.S. Operating Systems, Dr. Muharram Mansoorizadeh, spring 2010.\
		\item Teaching assistant, B.S. Computer Graphics, Dr. Mir Hossein Dezfoulian, autumn 2010.\
	\end{itemize}}

\cventry{2008--2010}{Teacher Assistant}{Hamedan University of Technology}{Hamedan, Iran}{}{Detailed achievements:%
	\begin{itemize}%
		\item Teaching assistant, B.S. Introduction to Programming, M.Sc. Hassan Bashiri, spring 2008.
		\item Teaching assistant, B.S. Advanced Programming, M.Sc. Hassan Bashiri, autumn 2008.
		\item Teaching assistant, B.S. Introduction to Assembly 80x86 Programming, M.Sc. Hatam Abdoli, spring 2009.
		\item Teaching assistant, B.S. Data Structures, Dr. Mir Hossein Dezfoulian, autumn 2009.
		\item Teaching assistant, B.S. Operating Systems, Dr. Muharram Mansoorizadeh, spring 2010.\
		\item Teaching assistant, B.S. Computer Graphics, Dr. Mir Hossein Dezfoulian, autumn 2010.\
	\end{itemize}}

\section{Experience}
\subsection{Vocational}
\cventry{2014--2016}{Student Job}{Fraunhofer IGD}{Darmstadt}{}{Computer graphics research and developments.\newline{}%
Detailed achievements:%
	\begin{itemize}%
	\item participating in VELaSSCo EC project development;
	\item Higher-order primitive ray-tracer implemented in Intel Embree and NVIDIA OptiX.
	\item Virtual reality development with LEAP Motion and Oculus SDK.
	\end{itemize}}
\cventry{2014--2014}{Research Assistant}{TUM's TUM's Foerdertechnik Materialfluss Logistik (FML) group}{Garching bei M\"unchen}{}{Detailed achievements:%
	\begin{itemize}%
		\item Working on 3D-Visualization of electromagnetic field strength distribution.
	\end{itemize}}
\cventry{2013--2014}{Guided Research}{TUM's Prof. Westermann's chair (Computer Graphics and Visualization)}{Garching bei M\"unchen}{}{\textbf{Topic:} Measuring and Evaluating Impact of Ray Sorting Algorithms on Coherency of SIMDs in Voxel-Based Path Tracers}
%{}{Abstract:%
%	Ray-traced global illumination (GI) algorithms are becoming widespread in real-time applications and computer video games. Path tracing is a common rendering technique to render images with global illumination effect. Low performance of these algorithms, makes their usage limited. To speed up these algorithms, some acceleration hi- erarchy like kd-tree, BSP-tree, etc. is used. Typically, BVH-Trees are used to accelerate path-tracing algorithms. Recently, these algorithms are run in real time on CPUs and GPUs but the ray coherency after the first bounce becomes too low; As CPUs and GPUs use wide SIMD units, gaining high coherency on these units is very important. A coherency improvement mechanism can be used to restore the ray coherency. In this paper we are going to investigate the impact of ray sorting on execution coherency of processor’s SIMD units. Our measurements show that execution coherency is increased by sorting the secondary rays but the improvements in coherency values are not as much as we expected and was presented by other papers. The scenes which are rendered in our approach are voxelized and stored in Boundary Volume Hierarchy (BVH) acceleration hierarchy. Furthermore, the voxel-based path-tracer is used as global illumination technique.}
\cventry{2013--2014}{Research Assistant}{TUM's Prof. Navab's chair (Computer Aided and Medical Procedures \& Augmented Reality)}{Garching bei M\"unchen}{}{Detailed achievements:%
	\begin{itemize}%
		\item Working on OpenGL debugging tools.
		\item Implemnting advanced ray caster for volume rendering of medical data.
	\end{itemize}}
\cventry{2013--2013}{Practical Course}{TUM's Prof. Cremers's chair (Computer Vision)}{Garching bei M\"unchen}{}{Topic:%
	\textbf{GPU Programming in Computer Vision}. Implementing optical flow and super resolution algorithms on GPU using CUDA.}
\cventry{2012--2013}{Student Job}{Developer at MetaIO GMbH}{M\"unchen}{}{Detailed achievements:%
	\begin{itemize}%
		\item Developing different Metaio's Junaio browser channels using HTML5, JavaScript, PHP, and MetaIO creator.
		\item Developing a hair-coloring C++ module using Metaio SDK.
		\item Participating into development of a game using Unity.
		\item 3D content creation and adjustments for mobile AR scenarios using 3D Studio Max.
	\end{itemize}}
\cventry{2012--2013}{Practical Course}{TUM's Prof. Westermann's chair (Computer Graphics and Visualization)}{Garching bei M\"unchen}{}{Topic:%
	\textbf{Interactive Visual Data Analysis} by using Direct3D 11 and C++.}
\cventry{2012--2012}{Student Job}{Developer at Fortiss GMbH}{M\"unchen}{}{Detailed achievements:%
	\begin{itemize}%
		\item Implementing an interface using windows message passing API to update the automotive system visualization in Ciros studio.
	\end{itemize}}


\section{Honors, Awards, Fellowships}
\cvlistitem{TUM Scholarship for International Students, Summer 2013, Winter 2013-14, and Summer 2015}
\cvlistitem{1st Place in Local Hamedan Azad University ACM Programming Contest, Hamedan Azad University, 2010}
\cvlistitem{1st Place in Local Hamedan University of Technology ACM Programming Contest, Hamedan University of Technology, 2009}
\cvlistitem{2nd Place in Local Bu-Ali Sina Hamedan University ACM Programming Contest, Bu-Ali Sina University, 2007}

\section{Languages}
\cvitemwithcomment{English}{TOEFL iBT Score(2011): 85 (Reading: 25, Listening: 19, Speaking: 17, Writing: 24)}{}
\cvitemwithcomment{Persian}{Mother Language}{}
\cvitemwithcomment{German}{Elementary}{}

\section{Computer skills}
\cvitem{\textbf{Advanced in using}}{C/C++, CMake, OpenGL, Vulkan, OpenCL, GLSL shader programming, Qt, \textbf{Ray tracing libraries} (NVIDIA Optix, Intel Embree)}
\cvitem{\textbf{Love to use}}{Latex, Git, and Linux}
\cvitem{\textbf{Have used}}{DirectX and HLSL shader programming, CUDA programming, DOxygen Commenting, 3D object modeling and animation using 3D Studio Max, Game engines (Ogre, Irrlicht, Unity, UDK), Windows programming, 80x86 Assembly programming, Microsoft Foundation Classes (MFC),  Thrift C++ API, WinSocket Programming, Creating AR content using Metaio SDK, Metaio Creator, and HTML5+Javascript+PHP (not available anymore since Metaio is sold to Apple), Pascal}


%\section{References}
%\begin{cvcolumns}
%  \cvcolumn{Category 1}{\begin{itemize}\item Person 1\item Person 2\item Person 3\end{itemize}}
%  \cvcolumn{Category 2}{Amongst others:\begin{itemize}\item Person 1, and\item Person 2\end{itemize}(more upon request)}
%  \cvcolumn[0.5]{All the rest \& some more}{\textit{That} person, and \textbf{those} also (all available upon request).}
%\end{cvcolumns}

% Publications from a BibTeX file without multibib
%  for numerical labels: \renewcommand{\bibliographyitemlabel}{\@biblabel{\arabic{enumiv}}}% CONSIDER MERGING WITH PREAMBLE PART
%  to redefine the heading string ("Publications"): \renewcommand{\refname}{Articles}
%\nocite{*}
%\bibliographystyle{plain}
%\bibliography{publications}                        % 'publications' is the name of a BibTeX file

% Publications from a BibTeX file using the multibib package
%\section{Publications}
%\nocitebook{book1,book2}
%\bibliographystylebook{plain}
%\bibliographybook{publications}                   % 'publications' is the name of a BibTeX file
%\nocitemisc{misc1,misc2,misc3}
%\bibliographystylemisc{plain}
%\bibliographymisc{publications}                   % 'publications' is the name of a BibTeX file



%\clearpage\end{CJK*}                              % if you are typesetting your resume in Chinese using CJK; the \clearpage is required for fancyhdr to work correctly with CJK, though it kills the page numbering by making \lastpage undefined
\end{document}


%% end of file `template.tex'.
