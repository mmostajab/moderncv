%% start of file `template.tex'.
%% Copyright 2006-2015 Xavier Danaux (xdanaux@gmail.com).
%
% This work may be distributed and/or modified under the
% conditions of the LaTeX Project Public License version 1.3c,
% available at http://www.latex-project.org/lppl/.


\documentclass[11pt,a4paper,sans]{moderncv}        % possible options include font size ('10pt', '11pt' and '12pt'), paper size ('a4paper', 'letterpaper', 'a5paper', 'legalpaper', 'executivepaper' and 'landscape') and font family ('sans' and 'roman')

% moderncv themes
\moderncvstyle{casual}                             % style options are 'casual' (default), 'classic', 'banking', 'oldstyle' and 'fancy'
\moderncvcolor{blue}                               % color options 'black', 'blue' (default), 'burgundy', 'green', 'grey', 'orange', 'purple' and 'red'
%\renewcommand{\familydefault}{\sfdefault}         % to set the default font; use '\sfdefault' for the default sans serif font, '\rmdefault' for the default roman one, or any tex font name
%\nopagenumbers{}                                  % uncomment to suppress automatic page numbering for CVs longer than one page

% character encoding
%\usepackage[utf8]{inputenc}                       % if you are not using xelatex ou lualatex, replace by the encoding you are using
%\usepackage{CJKutf8}                              % if you need to use CJK to typeset your resume in Chinese, Japanese or Korean

% adjust the page margins
\usepackage[scale=0.75]{geometry}
%\setlength{\hintscolumnwidth}{3cm}                % if you want to change the width of the column with the dates
%\setlength{\makecvheadnamewidth}{10cm}            % for the 'classic' style, if you want to force the width allocated to your name and avoid line breaks. be careful though, the length is normally calculated to avoid any overlap with your personal info; use this at your own typographical risks...

% personal data
\name{Morteza}{Mostajab}
%\title{}                               % optional, remove / comment the line if not wanted
\address{Kasinostr. 24}{64293 Darmstadt}{Germany}% optional, remove / comment the line if not wanted; the "postcode city" and "country" arguments can be omitted or provided empty
\phone[mobile]{+49~(170)~548~5750}                   % optional, remove / comment the line if not wanted; the optional "type" of the phone can be "mobile" (default), "fixed" or "fax"
\email{mmostajab@gmail.com}                               % optional, remove / comment the line if not wanted
\homepage{www.mmostajab.com}                         % optional, remove / comment the line if not wanted
%\social[linkedin]{john.doe}                        % optional, remove / comment the line if not wanted
\social[twitter]{https://twitter.com/mmostajab}                             % optional, remove / comment the line if not wanted
\social[github]{https://github.com/mmostajab}                              % optional, remove / comment the line if not wanted
%\extrainfo{additional information}                 % optional, remove / comment the line if not wanted
\photo[64pt][0.4pt]{mypic.jpg}                       % optional, remove / comment the line if not wanted; '64pt' is the height the picture must be resized to, 0.4pt is the thickness of the frame around it (put it to 0pt for no frame) and 'picture' is the name of the picture file
\quote{Researcher and Fan of Computer Graphics and Visualization}                                 % optional, remove / comment the line if not wanted

% bibliography adjustements (only useful if you make citations in your resume, or print a list of publications using BibTeX)
%   to show numerical labels in the bibliography (default is to show no labels)
\makeatletter\renewcommand*{\bibliographyitemlabel}{\@biblabel{\arabic{enumiv}}}\makeatother
%   to redefine the bibliography heading string ("Publications")
%\renewcommand{\refname}{Articles}

% bibliography with mutiple entries
%\usepackage{multibib}
%\newcites{book,misc}{{Books},{Others}}
%----------------------------------------------------------------------------------
%            content
%----------------------------------------------------------------------------------
\begin{document}
%\begin{CJK*}{UTF8}{gbsn}                          % to typeset your resume in Chinese using CJK
%-----       resume       ---------------------------------------------------------
\makecvtitle

\section{Education}
\cventry{2012--2016}{Master of Computer Science}{Technische Universit\"at M\"unchen}{Munich}{}{
	\textbf{Specialization}: Computer graphics and visualization \newline{}
	\textbf{Thesis Title}: Real-time Streamsurface Computation\newline{}%
	\textbf{Supervisor}: Prof.Dr. Westermann\newline{}%
	\textbf{Advisors}:  Dr. Andreas Dietrich, Dr. Frank Michel\newline{}%
}  % arguments 3 to 6 can be left empty
\cventry{2006--2011}{Bacholer of Computer Engineering}{Hamedan University of Technology}{Hamedan, Iran}{}{
	\textbf{Specialization}: Computer hardware engineering \newline{}
	\textbf{Thesis Title}: Incorporating affective state of players in video games \newline{}
	\textbf{Supervisor}: Dr. Muharram Mansoorizadeh \newline{}
}
%{Grade: \textit{15.04} out of 20.0}
\cventry{2002--2006}{Pre-University and High School}{National Organization for Development of Exceptional Talents' Shahid Beheshti School}{Borujerd, Iran}{}{Major: Mathematics and physics}

\section{Research Interests}
\cvitem{}{Rendering techniques (ray tracing and rasterization)}
\cvitem{}{Virtual reality}
\cvitem{}{SciVis techniques}
\cvitem{}{Computer graphics and visualization }
\cvitem{}{Object oriented programming}

%\section{Master thesis}
%\cvitem{title}{\emph{Real-time streamsurface computation and rendering}}

%\cvitem{abstract}{Streamsurfaces are one of the powerful visualization tools, which are used to gain insight into characteristics and features of flow fields. In practice, streamsurfaces are approximated by triangulating adjacent pairs of integral curves, originating from a seeding line. The generation of integral curves bears quite some similarities to ray tracing algorithms used in physically based renderers. Although, the techniques used in ray tracing may not have good performance in the streamline computation context due to their different computational nature, they can be optimized for streamline computation by introducing some modifications. In this master thesis, I present my work on accurate streamsurface computation and rendering in real-time, by exploiting the scalability and portability features of parallel architectures in heterogeneous computing, and utilizing concepts from physically based rendering. To improve the efficiency, I use a scheduler to divide the streamsurface computation and rendering tasks on different devices proportional to their computation powers. Additionally, I apply and evaluate different acceleration structures and the concepts of caching to improve the efficiency and utilization of streamsurface generation on modern GPUs and CPUs to achieve real-time results. Furthermore, the possible impact of applying ray-packing and ray-sorting to the streamline computation is investigated.}

%\section{Bachelor thesis}
%\cvitem{title}{\emph{Incorporating affective states of players in video games}}
%\cvitem{supervisor}{Dr. Muharram Mansoorizadeh}
%\cvitem{abstract}{Video games are very popular, these days. They use different ways to interact with users. Keyboards, mouse, Joysticks, and recently camera are among the popular interaction devices. While technology involved in game industry develops new interactive devices to get user's interactions that capture lots of player's inputs, the games just use player's voluntary interactions. However, spontaneous behaviors of players are of much value in playing games, the players do not use involuntary interactions. Among several available representations for affects, two-dimensional continuous Activation/Evaluation(Valence) space has been adopted. We can determine the player's preferences in the environment by comparing his or her playing to the player's valence and activity in real-time. These information can be used to modify the game rules and environment to make it more attractive for players and create a unique experience during each gameplay session. (Score:20/20).}

\section{Publications}

\cvitem{}{\textbf{CSG Ray Tracing Revisited-Visualizing Massive Models} \newline{}
by Morteza Mostajab, Andreas Dietrich, Thomas Gierlinger, Frank Michel, Andre Stork (The second draft is ready. It is being prepared for submission).}

\cvitem{}{\textbf{Real-Time Stream Surface Computation and Rendering Utilizing Heterogeneous Computing} \newline{}
	by Morteza Mostajab, Andreas Dietrich, Thomas Gierlinger, Frank Michel, Andre Stork (The first draft is ready. It is being prepared for submission).}

\newpage

\section{Work Experiences}

\cventry{2016--Present}{Researcher and developer}{Fraunhofer IGD}{Darmstadt}{}{Research Area: Rendering Techniques, and Query-Based Visualization \newline{} %
	Project:%
	\begin{itemize}
		\item VELaSSco (Visualization For Extremely Large-Scale Scientific Computing) EC project (\href{htpp://www.velassco.eu}{VELaSSco.eu}). 
	\end{itemize}}


\section{University Projects and Research}
\cventry{2014--2016}{Student researcher and developer}{Fraunhofer IGD}{Darmstadt}{}{Related to computer graphics research and developments.
	\begin{itemize}%
	\item Involving into VELaSSCo EC project development.
	\item Higher-order primitive ray tracer implemented in Intel Embree and NVIDIA OptiX.
	\item Virtual reality development with LEAP Motion and Oculus SDK.
	\end{itemize}}
\cventry{2014--2014}{Student researcher and developer}{TUM's TUM's Foerdertechnik Materialfluss Logistik (FML) group}{Garching bei M\"unchen}{}{
	\begin{itemize}%
		\item Working on 3D visualization of electromagnetic field strength distribution.
	\end{itemize}}
\cventry{2013--2014}{Guided Research}{TUM's Prof. Westermann's chair (Computer Graphics and Visualization)}{Garching bei M\"unchen}{}{\textbf{Topic:} Measuring and Evaluating Impact of Ray Sorting Algorithms on Coherency of SIMDs in Voxel-Based Path Tracers}
%{}{Abstract:%
%	Ray-traced global illumination (GI) algorithms are becoming widespread in real-time applications and computer video games. Path tracing is a common rendering technique to render images with global illumination effect. Low performance of these algorithms, makes their usage limited. To speed up these algorithms, some acceleration hi- erarchy like kd-tree, BSP-tree, etc. is used. Typically, BVH-Trees are used to accelerate path-tracing algorithms. Recently, these algorithms are run in real time on CPUs and GPUs but the ray coherency after the first bounce becomes too low; As CPUs and GPUs use wide SIMD units, gaining high coherency on these units is very important. A coherency improvement mechanism can be used to restore the ray coherency. In this paper we are going to investigate the impact of ray sorting on execution coherency of processor’s SIMD units. Our measurements show that execution coherency is increased by sorting the secondary rays but the improvements in coherency values are not as much as we expected and was presented by other papers. The scenes which are rendered in our approach are voxelized and stored in Boundary Volume Hierarchy (BVH) acceleration hierarchy. Furthermore, the voxel-based path-tracer is used as global illumination technique.}
\cventry{2013--2014}{Student researcher and developer}{TUM's Prof. Navab's chair (Computer Aided and Medical Procedures \& Augmented Reality)}{Garching bei M\"unchen}{}{
	\begin{itemize}%
		\item Working on OpenGL debugging tools.
		\item Implemnting advanced ray caster for volume rendering of medical data.
	\end{itemize}}
\cventry{2013--2013}{Practical Course}{TUM's Prof. Cremers's chair (Computer Vision)}{Garching bei M\"unchen}{}{\textbf{Topic}:%
	GPU Programming in Computer Vision. Implementing optical flow and super resolution algorithms on GPU using CUDA.}
\cventry{2012--2013}{Student researcher and developer}{Metaio GmbH}{M\"unchen}{}{
	\begin{itemize}%
		\item Developing different Metaio's Junaio browser channels using HTML5, JavaScript, PHP, and Metaio creator.
		\item Developing a hair-coloring C++ module using Metaio SDK.
		\item Participating into development of a game using Unity.
		\item 3D content creation and adjustments for mobile AR scenarios using 3D Studio Max.
	\end{itemize}}
\cventry{2012--2013}{Practical Course}{TUM's Prof. Westermann's chair (computer graphics and visualization)}{Garching bei M\"unchen}{}{\textbf{Topic}:%
	Interactive Visual Data Analysis using Direct3D 11 and C++.}
\cventry{2012--2012}{Student researcher and developer}{Fortiss GmbH}{M\"unchen}{}{
	\begin{itemize}%
		\item Implementing an interface using windows message passing API to update the automotive system visualization in Ciros studio.
	\end{itemize}}

\newpage{}

\section{Teaching}

\cventry{2016}{Seminar Course Supervision}{Technische Universitaet Darmstadt}{Germany}{}{Topics:
	% http://www.gris.tu-darmstadt.de/teaching/sempract/ss16/scivis/index.de.htm
	\begin{itemize}
		\item 
		Apex Point Map for Constant-Time Bounding Plane Approximation by Laine, Samuli. Karras, Tero.
		\item SIMD Parallel Ray Tracing of Homogeneous Polyhedral Grids by Rathke, Brad; Wald, Ingo; Chiu, Kenneth; Brownlee, Carson.
	\end{itemize}}
	
	\cventry{2008--2010}{Teacher Assistant}{Hamedan University of Technology}{Hamedan, Iran}{}{
		\begin{itemize}%
			\item Teaching assistant, Introduction to Programming, Spring 2008.
			\item Teaching assistant, Advanced Programming, Autumn 2008.
			\item Teaching assistant, Introduction to Assembly 80x86 Programming, Spring 2009.
			\item Teaching assistant, Data Structures, Autumn 2009.
			\item Teaching assistant, Operating Systems, Spring 2010.\
			\item Teaching assistant, Computer Graphics, Autumn 2010.\
		\end{itemize}}

\section{Honors, Awards, Fellowships}
\cvlistitem{Winning TUM's Scholarship for International Students in Summer 2013, Winter 2013-14, and Summer 2015.}
\cvlistitem{1st Place (2009 and 2010), 2nd Place (2007) in Local Hamedan, Iran ACM Programming Contests}

\section{Languages}
\cvitemwithcomment{English}{Professional working proficiency}{}
\cvitemwithcomment{German}{Elementary}{}
\cvitemwithcomment{Persian}{Native}{}

\section{Computer skills}
\cvitem{\textbf{Programming Languages}}{C/C++, and Python.}
\cvitem{\textbf{Frameworks and Libraries}}{OpenGL, OpenCL, GLSL shader programming, Qt, Ray tracing libraries (NVIDIA Optix, Intel Embree), Vulkan, Direct3D 11 and HLSL shader programming, and CUDA programming}
\cvitem{\textbf{Operating Systems}}{Windows, and Linux.}
\cvitem{\textbf{Version Control}}{Git, SVN, and Perforce.}
\cvitem{\textbf{Document- ation}  }{Latex, and MarkDeep.}
\cvitem{\textbf{3D Software Package}}{3D Studio Max.}

\newpage{}

\section{References}
\cvlistitem{\textbf{
		Prof. Dr. Ruediger Westermann} \newline{}
		Homepage: \href{http://wwwcg.in.tum.de/group/persons/westermann.html}{ http://wwwcg.in.tum.de/group/persons/westermann.html} \newline{}
		E-mail: \href{mailto:westermann@tum.de}{westermann@tum.de} \newline{}
		}
\cvlistitem{\textbf{
		Dr. Andreas Dietrich} \newline{}
		E-mail: \href{mailto:andi.dietrich@googlemail.com}{andi.dietrich@googlemail.com} \newline{}
	}

% Publications from a BibTeX file without multibib
%  for numerical labels: \renewcommand{\bibliographyitemlabel}{\@biblabel{\arabic{enumiv}}}% CONSIDER MERGING WITH PREAMBLE PART
%  to redefine the heading string ("Publications"): \renewcommand{\refname}{Articles}
%\nocite{*}
%\bibliographystyle{plain}
%\bibliography{publications}                        % 'publications' is the name of a BibTeX file

% Publications from a BibTeX file using the multibib package
%\section{Publications}
%\nocitebook{book1,book2}
%\bibliographystylebook{plain}
%\bibliographybook{publications}                   % 'publications' is the name of a BibTeX file
%\nocitemisc{misc1,misc2,misc3}
%\bibliographystylemisc{plain}
%\bibliographymisc{publications}                   % 'publications' is the name of a BibTeX file



%\clearpage\end{CJK*}                              % if you are typesetting your resume in Chinese using CJK; the \clearpage is required for fancyhdr to work correctly with CJK, though it kills the page numbering by making \lastpage undefined
\end{document}


%% end of file `template.tex'.
